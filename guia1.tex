\documentclass{article}
\usepackage[a4paper, total={6.2in, 10in}]{geometry}
\usepackage{enumitem}
\usepackage{mathtools}
\usepackage{graphicx}
\usepackage{amssymb}
\begin{document}

\title{Práctica 1: Lenguajes}
\date{2do cuatri 2024 - si encontrás algún error mi tg es @kztqz}

\renewcommand{\baselinestretch}{1.5} 

\maketitle

\section*{Ejercicio 1}{Sea $\Sigma$ = \{\textit{a, b}\} un alfabeto. Hallar:
\\
\\
\centerline{$\Sigma^0,\ \ \ \Sigma^1,\ \ \ \Sigma^2,\ \ \ \Sigma^*,\ \ \ \Sigma^+,\ \ \ |\Sigma|,\ \ \ |\Sigma^0|$}

\begin{itemize}
    \item $\Sigma^0 = \{\lambda\}$
    \item $\Sigma^1 = \Sigma = \{a, b\}$
    \item $\Sigma^2 = \{aa, ab, ba, bb\}$
    \item $\Sigma^* = \bigcup\limits_{i \ge 0}\Sigma^i = \{\lambda, a, b, aa, ab, ba, bb\}$
    \item $\Sigma^+ = \bigcup\limits_{i \ge 1}\Sigma^i = \{a, b, aa, ab, ba, bb\}$
    \item $|\Sigma| = 2$
    \item $|\Sigma^0| = 0$
\end{itemize}

\section*{Ejercicio 2}{Decidir si, dado $\Sigma$ = \{\textit{a, b}\} vale:}
\\
\\
\centerline{$\lambda\in\Sigma,\ \ \ \lambda\subseteq\Sigma,\ \ \ \lambda\in\Sigma^+,\ \ \ \lambda\in\Sigma^*,\ \ \ \Sigma^0=\lambda,\ \ \ \Sigma^0=\{\lambda\}$}

\begin{itemize}
    \item $\lambda\in\Sigma \rightarrow {Falso}$ 
    \item $\lambda\subseteq\Sigma \rightarrow {Falso}$ 
    \item $\lambda\in\Sigma^+ \rightarrow {Falso}$ 
    \item $\lambda\in\Sigma^* \rightarrow {Verdadero}$ 
    \item $\Sigma^0=\lambda \rightarrow {Falso}$ 
    \item $\Sigma^0=\{\lambda\} \rightarrow {Verdadero}$ 
\end{itemize}

\section*{Ejercicio 3}{Sea \textit{$\alpha = abb$} una cadena. Calcular:}
\\
\\
\centerline{$\alpha^0,\ \ \ \alpha^1,\ \ \ \alpha^2,\ \ \ \alpha^3,\ \ \ \prod_{k=0,...,3}{\alpha^k = \alpha^0.\alpha^1.\alpha^2.\alpha^3},\ \ \ \alpha^r$}

\begin{itemize}
    \item $\alpha^0 = \lambda$
    \item $\alpha^1 = \textit{abb}$
    \item $\alpha^2 = \textit{abb}{.}\textit{abb}$
    \item $\alpha^3 = \textit{abb}{.}\textit{abb}{.}\textit{abb} = \textit{abbabbabb}$
    \item $\prod_{k=0,...,3}{\alpha^k = \alpha^0.\alpha^1.\alpha^2.\alpha^3} = \lambda.\textit{abb}{.}\textit{abbabb}{.}\textit{abbabbabb} = \textit{abbabbabbabbabbabb}$
    \item $\alpha^r = (\textit{abb})^r = \textit{bba}$
\end{itemize}

\section*{Ejercicio 4}{{Sean las cadenas \textit{$\alpha = abb$} y \textit{$\beta = acb$}. Calcular:}
\\
\\
\centerline{$\alpha\beta,\ \ \ (\alpha\beta)^r,\ \ \ \beta^r,\ \ \ \beta^r\alpha^r,\ \ \ \lambda\alpha,\ \ \ \lambda\beta, \ \ \ \alpha\lambda\beta, \ \ \ \alpha^2\lambda^3\beta^2$}

\begin{itemize}
    \item $\alpha\beta = \textit{abbacb}$
    \item $(\alpha\beta)^r = (\textit{abbacb})^r = \textit{bcabba}$
    \item $\beta^r = (\textit{acb})^r = \textit{bca}$
    \item $\beta^r\alpha^r = (\textit{acb})^r(\textit{abb})^r = \textit{bcabba}$
    \item $\lambda\alpha = \alpha = \textit{abb}$
    \item $\lambda\beta = \beta = \textit{acb}$
    \item $\alpha\lambda\beta = \alpha\beta = \textit{abbacb}$
    \item $\alpha^2\lambda^3\beta^2 = \alpha^2\beta^2 = \textit{abbabbacbacb}$
\end{itemize}

\section*{Ejercicio 5}{Dado un alfabeto $\Sigma$, sean \textit{x},\textit{y} $\in \Sigma$ y $\alpha,\beta \in \Sigma^*$. Demostrar que:}
\begin{enumerate}[label=\alph*.,font=\itshape]
    \item $|x.(y.\alpha)| = 2 + |\alpha|$
    \item $|\alpha^r|=|\alpha|$
    \item $|\alpha x \beta| = |x \alpha \beta|$
    \item $|\alpha . \alpha = 2 |\alpha|$
    \item $(\alpha . \beta)^r = \beta^r . \alpha^r$
    \item $(\alpha^r)^r = \alpha$
    \item $(\alpha^r)^n = (\alpha^n)^r$
\end{enumerate}

{($|\alpha|$ indica la longitud de la cadena $\alpha)$.}
\\
\\
{Para resolver estos ejercicios va a ser útil tener a mano estas def / demos:}

\renewcommand\labelenumi{(\theenumi)}
\begin{enumerate}
    \item {Definición recursiva de la longitud:}
    \\
    \\
    \centerline{{$|\lambda| = 0$}}
    \\
    \\
    \centerline{{$|x.\alpha| = 1 + |\alpha|$}}

    \item {Propiedad: $|\alpha + \beta| = |\alpha|.|\beta|$. Demo por inducción estructural sobre $\alpha$:}
    \\
    \\
    \centerline{1. Si $\alpha = \lambda$:}
    \\
    \\
    \centerline{$|\lambda.\beta| = |\beta| = 0 + |\beta| = |\lambda| + |\beta|$}
    \\
    \\
    \centerline{2. Si $\alpha = x.\alpha'$, suponemos que vale para $\alpha'$, y:}
    \\
    \\
    \centerline{$|(x.\alpha').\beta| = |x.(\alpha'.\beta)|$ \ \textbf{(def. $\alpha'$)}}
    \\
    \\
    \centerline{ \ \ \ \ \ \ \ \ \ \ \ \ \ \ \ \ \ \ \ \ \ $= 1 +|\alpha'.\beta|$  \ \textbf{(def. longitud)}}
    \\
    \\
    \centerline{$ \ \ \ \ \ \ \ \ \ \ \ \ \ \ \ \ \ \ \ = 1 +|\alpha'| + |\beta|$ \ \textbf{(hip. ind)}}
    \\
    \\
    \centerline{$ \ \ \ \ \ \ \ \ \ \ \ \ \ \ \ \ \ \ \ \ \ \ \ \ = |x.\alpha'| + |\beta|$ \ \textbf{(def. longitud)}}
    \\
    \\
    \centerline{$ \ \ \ \ \ \ \ \ \ \ \ = |\alpha| + |\beta|$ \ \textbf{(def. $\alpha'$)}}    

    \item {Definición recursiva de la reversa:}
    \\
    \\
    \centerline{$\lambda^r = \lambda$}
    \\
    \\
    \centerline{$(x.\alpha)^r = \alpha^r.x$}
\end{enumerate}


\begin{enumerate}[label=\alph*.,font=\itshape]
    \item {Quiero demostrar que $|x.(y.\alpha)| = 2 + |\alpha|$.
    \\
    \\
    {Como $x \in \Sigma$, por la definición recursiva de longitud, puedo sacarlo de la operación longitud \\ \\ de la cadena y sumar un 1. Lo mismo hago para $y \in \Sigma$.}
    \\
    \\
$|x.(y.\alpha)| \ \ {\stackrel{\mathclap{\normalfont\mbox{(1)}}}{=}} \ \ 1 + |y.a| \ \ {\stackrel{\mathclap{\normalfont\mbox{(1)}}}{=}} \ \ 1 + 1 + |\alpha| = 2 + |\alpha|$}
    \\ 
    \item {Quiero demostrar que $|\alpha^r|=|\alpha|$. Pruebo por inducción: \\ \\
    \centerline{1. Caso base (si $\alpha = \lambda$):}
    \\
    \\
    \centerline{$|\lambda|^r{ \ \stackrel{\mathclap{\normalfont\mbox{(3)}}}{=}} \ |\lambda|$}
    \\
    ...
    \\
    \\
    continuará}
\end{enumerate}

\section*{Ejercicio 6}{Dar ejemplos de cadenas que pertenezcan a los siguientes lenguajes:}
\begin{enumerate}[label=\alph*.,font=\itshape]
    \item {$\mathcal{L} = \{a^nb^n \ | \ n \geq 0\}$}
    \item {$\mathcal{L} = \{a^nb^n \ | \ n \geq 1\}$}
    \item {$\mathcal{L} = \{a^nb^m \ | \ n \geq 1 \land m \geq 1\}$}
    \item {$\mathcal{L} = \{a^nb^m \ | \ n \geq 1 \land m \geq 0\}$}
    \item {$ \mathcal{L} = \{ a^n(ac)^p(bab)^q \ | \ n \geq 0 \land q = p + 2 \land p \geq 1$ \}}
    \item {$\mathcal{L} = \{ a, b \}^3 \ \bigcap \ \Lambda $}
    \item {$\mathcal{L} = \{ \alpha\alpha^r \ | \ \alpha \in \{ a,b \} ^+\}$}
    \item {$\mathcal{L} = \{ \alpha \in \{ a,b \} ^+ \ | \ \alpha = \alpha^r \}$ \\ \\}
\end{enumerate}

\begin{enumerate}[label=\alph*.,font=\itshape]
    \item {$\mathcal{L} = \{a^nb^n \ | \ n \geq 0\}$
    \\
    \\
    {En castellano: todas las cadenas de $\mathcal{L}$ tienen la misma cantidad de $a$ y $b$ y $\lambda \in \mathcal{L}$.}
    \\
    \\
    Ejemplos: \{$\lambda, ab, aabb, aaabbb$\}}
    \item {$\mathcal{L} = \{a^nb^n \ | \ n \geq 1\}$
    \\
    \\    
    {En castellano: las cadenas de $\mathcal{L}$ tienen al menos una $a$ y una $b$ (por lo tanto, $\lambda \notin \mathcal{L}$) y la cantidad de apariciones de $a$ es igual a la cantidad de apariciones de $b$.}
    \\
    \\
    Ejemplos: \{$ab, aabb, aaabbb$\}}
    \item {$\mathcal{L} = \{a^nb^m \ | \ n \geq 1 \land m \geq 1\}$
    \\
    \\
    {En castellano: las cadenas de $\mathcal{L}$ tienen al menos una $a$ y una $b$ (por lo tanto, $\lambda \notin \mathcal{L}$) y la cantidad de apariciones de $a$ no es necesariamente igual a la cantidad de apariciones de $b$.}
    \\
    \\
    Ejemplos: \{$ a, b ,ab, aa, bb, baa, aba$\}}
    \item {$\mathcal{L} = \{a^nb^m \ | \ n \geq 1 \land m \geq 0\}$
    \\
    \\
    {En castellano: las cadenas de $\mathcal{L}$ tienen al menos una $a$ (por lo tanto, $\lambda \notin \mathcal{L}$) y la cantidad de apariciones de $a$ no es necesariamente igual a la cantidad de apariciones de $b$.}
    \\
    \\
    Ejemplos: \{$a, ab, aab, aaa, abab$\}}
    \item {$\mathcal{L} = \{ a^n(ac)^p(bab)^q \ | \ n \geq 0 \land q = p + 2 \land p \geq 1 \}$
    \\
    \\
    {En castellano: como $p \geq 1$, las cadenas de $\mathcal{L}$ tienen al menos una aparición de $ac$. Además, como $q = p + 2$, tengo que $q \geq 3$, todas las cadenas de $\mathcal{L}$ tienen al menos tres apariciones de $bab$. La cantidad de apariciones de $bab$ depende de la cantidad de apariciones de $ac$. Las cadenas de $\mathcal{L}$ pueden no tener apariciones de $a$ y estas no dependen ni de $p$ ni de $q$.
    \\
    \\
    Ejemplos: \{$acbabbabbab, acacbabbabbabbab, aacbabbabbab$\}}}
    \item {$\mathcal{L} = \{ a, b \}^3 \ \bigcap \ \Lambda\}$
    \\
    \\
    {En castellano: ? }
    \\
    \\
    Ejemplos: \{$ ? $\}}
    \item {$\mathcal{L} = \{ \alpha\alpha^r \ | \ \alpha \in \{ a,b \} ^+ \} $
    \\
    \\
    {En castellano: todas las cadenas de $\mathcal{L}$ se forman concatenando una cadena $\alpha \in \{a,b\}^+$ con su reversa. Mirando el + de $\{a,b\}^+$ vemos que $\lambda \notin \mathcal{L}$. Como todos las cadenas en $\mathcal{L}$ son de la forma una cadena de longitud igual o mayor a 1 concatenada con su reversa (que tiene la misma longitud, mirar ejercicio 5) todas tienen longitud igual o mayor a 2.}
    \\
    \\
    Ejemplos: \{$aa, bb, abba, baab, aabbaa, babaabab$\}}

    
\end{enumerate}



\end{document}